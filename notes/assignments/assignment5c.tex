\documentclass{article}
\input{preamble.tex}

\begin{document}
    \section{Explanation of GroupSwap}

    The GroupSwap algorithm is an algorithm used to swap the values of two variables using arithmetics. It will behave differently depending on the semantics we use. 


    \begin{definition}
        Definition of groupswap: 
        \begin{lstlisting}
    procgroupswap =
      Proc
        "groupswap"
        [("x", (Upd, "Integer")), ("y", (Upd, "Integer"))]
        ( Sequence
            [ Assign "y" (Plus (Var "x") (Var "y")),
              Assign "x" (Minus (Var "y") (Var "x")),
              Assign "y" (Minus (Var "y") (Var "x"))
            ]
        )
        \end{lstlisting}

        Which translates to the following mathematical expressions

        \begin{align*}
            y &= x + y \\
            x &= y - x \\
            y &= y - x \\
        \end{align*}
    \end{definition}

    \subsection{Reference Semantics}
    Using reference semantics, the algorithm will behave in the following way:
    \medskip

    Given that \( x \) and \( y \) both point to the same variable \( a = 12 \), we obtain the following:

    \begin{align*}
        y &= 12 + 12 && x = 24 \quad y = 24 \\
        x &= 24 - 24 && x = 0 \quad y = 0 \\
        y &= 0 - 0 && x = 0 \quad y = 0\\
    \end{align*}

    \subsection{Copy in / Copy out semantics}

    Using copy in / copy out semantics, the algorithm works in the following way:
    \medskip

    Again, given two variables \( x = 12 \) and \( y = 24 \), we obtain the following

    \begin{align*}
        y &= 12 + 24 && x = 12 \quad y = 36 \\
        x &= 36 - 12 && x = 24 \quad y = 36 \\
        y &= 36 - 24 && x = 24 \quad y = 12\\
    \end{align*}

    \subsection{The Difference}

    As we can see, using the different semantics make the algorithm return different results. This is due to the fact that when we are using reference semantics, \( x \) and \( y \) both point to the same address in memory, thus both are making changes to the value stored at this address. When using copy in / copy out semantics, the variables each get their own memory address, which means that they won't change the same value. Using a language with support for pointers, such as C++, we can see this effect in action, demonstrated below

    \begin{eg}
        
        Using reference semantics, we write the following code:
        \begin{lstlisting}
    int main() {
        int a = 10;
        int *x = &a;
        int *y = &a;

        std::cout << "Before swap:" << std::endl;
        std::cout << "x = " << *x << std::endl; // x = 10
        std::cout << "y = " << *y << std::endl; // y = 10

        // Swap values pointed to by x and y
        *y = *x + *y;
        *x = *x - *y;
        *y = *x - *y;

        std::cout << "After swap:" << std::endl;
        std::cout << "x = " << *x << std::endl; // x = 0
        std::cout << "y = " << *y << std::endl; // y = 0
        return 0;
    }
        \end{lstlisting}

    Using copy in / copy out semantics, we can write the following code:


    \begin{lstlisting}
    int main() {
        int a = 10;
        int b = 15;
        int x = a;
        int y = b;

        std::cout << "Before swap:" << std::endl;
        std::cout << "x = " << x << std::endl;  // x = 10
        std::cout << "y = " << y << std::endl;  // y = 15

        // Swap values of x and y
        y = x + y;
        x = x - y;
        y = x - y;

        std::cout << "After swap:" << std::endl;
        std::cout << "x = " << x << std::endl;  // x = 15
        std::cout << "y = " << y << std::endl;  // y = 10
        return 0;
    }
    \end{lstlisting}
    \end{eg}



\end{document}


