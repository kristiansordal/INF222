\documentclass{article}
\input{preamble.tex}

\title{Assignment 2, INF222, Spring 2023}
\begin{document}
    \maketitle
    \newpage
    \tableofcontents
    \newpage

    \section{Top 500}
    \begin{enumerate}
        
        \item The top 500 list for June 2013, and even more so, for June 2016, created a wave of surprise for the “western world”. What was this surprise, and what were the political
implications?

    \medskip

    \begin{center}
        \fbox{
            \parbox{0.95\textwidth}{
               In the the top 500 list for June 2013, China's National University of Defense Technology topped the list with their Tianhe-2 system. They also in June 2016 held one of the top spots in the Green 500 list, which measures performance to power. The systems they had created contained a number of chinese produced and manufactured components, meaning that the United States were no longer the only country with the capability to produce the components required to create supercomputers.
            }
        }
    \end{center}
    \medskip

\item . At \texttt{https://top500.org/statistics/perfdevel/} there are some diagrams showing the
exponential growth of supercomputing power since the early 1990s. Notice how the
performance increase has slowed down since 2015. Can you find a few factors explaining
the slowdown?

\medskip

\begin{center}
    \fbox{
        \parbox{0.95\textwidth}{
            One of the key factors for the slowdown of increase in computer performance is due to the fact that fitting more transistors onto a chip is getting more and more difficult. This is due to the size of the transistors is already incredibly tiny, at just 5nm. When we get to such small sizes, or even smaller, quantum effects are at play which can impede computer / transistor performance.
        }
    }
\end{center}
\medskip

    \end{enumerate}

\end{document}
